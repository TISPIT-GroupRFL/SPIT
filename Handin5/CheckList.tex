\documentclass[Main]{subfiles}

\begin{document}

\section{Checklist}

This section provides a checklist approach to inspect an SRS. The questions used in this section are relevant questions from the requirements verification checklist used by \textit{System Development Life Cycle process (SDLC)}.

\subsection{Correct}
A Requirements Specification is correct if, and only if, every requirement stated therein is one that the system/software shall meet.

\begin{longtable}{p{8cm} | p{7.3cm}}
\textbf{Criteria} & \textbf{Yes / No / NA} \\ \hline

Is the expected response time, from the user’s point of view, specified for all necessary operations? & No. Only specified for one operation (Req ID 20140228-212400) \\ \hline

Are other timing considerations specified, such as processing time, data transfer, and throughput? & No. \\ \hline

Are all the tasks to be performed by the system/software specified? & No. Handling new joiners on the day is not clearly specified. \\ \hline

Does each task specify the data/information content used in the task and the data/information content resulting from the task? & No. 
\begin{itemize}
\item Req ID 20140301-103401 - List printing - Does not specify what data to be printed for each person.
\item Req ID 20140301-103403 - Adding employee in PIS - Relevant data for an employee not specified.
\end{itemize} \\ \hline

Are the physical security requirements specified? & N/A \\ \hline
Are the operational security requirements specified? & Yes \\ \hline

Is the reliability of the system/software specified, including the consequences of failure, vital information protected from failure, error detection, and recovery? & Yes. \\ \hline

Are acceptable trade-offs between competing attributes specified, for example, between robustness and correctness? & No. \\ \hline

Are internal interfaces, such as software and hardware, defined? & N/A. \\ \hline

Are external interfaces, such as users (e.g., people), software, and hardware defined? & Partially. \begin{itemize}
\item TIS/PIS interface does not specify data/format.
\item Food/drinks ordering and room reservation were mentioned in section 2.1 "Product perspective" but not specified.
\end{itemize} \\ \hline

Is the definition of success included? Of failure? & Yes. Availability required is specified in Req ID 20140228-213700. \\ \hline

Is each requirement relevant to the problem and its solution? & Yes. However, Food/drinks ordering and room reservation were mentioned in section 2.1 "Product perspective" but have not been requested by the customer. \\ \hline

\end{longtable}

\subsection{Unambiguous}
A Requirements Specification is unambiguous if, and only if, every requirement stated therein has only one interpretation.

\begin{longtable}{p{8cm} | p{7.3cm}}
\textbf{Criteria} & \textbf{Yes / No / NA} \\ \hline

Are the requirements specified clearly enough to be turned over to an independent group for implementation and still be understood? & No. The SRS is missing context (background, business goals, vision). \\ \hline

Are functional requirements separated from non-functional? & Yes, however the non-functional requirements are sorted by engineering roles - why? What significance does this have? \\ \hline

Are requirements specified in a concise manner that avoids the likelihood of multiple interpretations? & Partially. Some requirements are ambiguous, but those are clearly marked with comments stating what is TBD.
\begin{itemize}
\item Req ID 20140301-103402 - Badge printing - "print" is an ambiguous term. PDF is mentioned, but not in the requirement.
\item Req ID 20140301-124000 - Testing the web interface - What interface?
\item Req ID 20140301-124500 - The TBD comment poorly worded. What does it mean?
\item Req ID 20140301-124900 - Quality engineer - The requirement text is unclear, should the development use an agile process? 
\end{itemize} \\ \hline


\end{longtable}

\subsection{Complete}
A Requirements Specification is complete if, and only if, it includes the following elements:
\begin{itemize}
\item All significant requirements, whether relating to functionality, performance, design constraints, attributes, or external interfaces.
\item Definitions of the responses of the system/software to all realizable classes of input data in all realizable classes of situations.
\item Descriptive labels for and references to all figures, tables, and diagrams in the Requirements Specification and definition of all terms and units of measure.

\end{itemize}


\end{document}