\documentclass[Main]{subfiles}

\begin{document}

\chapter{Reflection}

The review from Group 5 is useful in many ways since it has highlighted some inconsistencies and areas for improvement.
\\
\\
Below is listed a few:

\begin{itemize}

\item A document ID has been added to all pages, so new versions can't be mixed with old versions.

\item A requirement ID has been added to all sections in the document.

\item A list of abbreviation is added.

\item Corrected text regarding maintaining

\item Bad reference updated 

\end{itemize}
This is some of the feedback that are actually useful for the SRS.

Listed was also that the 4+1-model should be used (which it should not, since this belongs in the Design document for the system), and the requirement's IDs should be "grep-friendly", here meaning it should be possible to search for them.
\\
\\
Comparing Task-Description vs. the traditional way of writing the customer's requirement, Task-Description seems easier to both write and understand.
Use cases normaly describe some interaction between some actor and the system under development.
An actor can be both a person and some other system.
On the other hand task's focueses on the complete scope of the task and leave it for later to decide who should do what and how.

By using tasks to describe what should be developed, architects can focus on what should be developed and not how it should be done, this is better left in the design phase of the project.

Focusing on what to develop is a good disposition because a too early focus on how to technically achiving something can limit ideas that could be better long term.

The reviewer was not in doubt of what should happen, or what the outcome would be of this given operation.

\end{document}
