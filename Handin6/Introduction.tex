\documentclass[Main]{subfiles}

\begin{document}


\section{Introduction}
%This part shall contain the number of the group and list its members and introduce the reader to the context of the informal specification of Handin4 (around 1 page).


This paper is written by Group 6, consisting of Rasmus Bækgaard and Lasse B. Pedersen, written for the course TISPIT at Aarhus University.


During the course a series of papers has been written on different topics regarding writing specifications for IT-Systems.
One of them, Handin 4 \parencite{HI4}, is written about a system for the course on Aarhus University, ITDMAT. 
The course requires the student to make several handins, send them to specific students in a rotation, review them and send the review to the teacher and the author of the handin.
Many things can go wrong in this rotation for the student, but far worse is the handling of about 100 emails the teacher receive with comments of each exercise.

To avoid this a system is needed to take care of all the handins, comments, distribution of reviewers and reviewees, and give the teacher an overview for who has passed and who has not.
\\
\\
The specification is written in the style of Søren Lauesen's Task-Description method, \parencite{Task}, and explains a new way of interpreting the customer's needs for a system.
When writing in this style, the customer gets a better overview of what will be done, where loopholes can be found and what can be expected of the system.
\\
\\
The document was written after a template Søren Lauesen has specified with all sections and explanations of what should be done writing such a paper.
First of all, this is a paper for the supplier.
That means it will start out with a introduction of how the document is to be read -- examples of how a task is build up \parencite[5-6]{HI4}, what was before and what will become after the system has been implemented, \parencite[4]{HI4}.

After this, the supplier will be introduced to what the document is written with thoughts in mind of -- the High-level demands, aka. the business goal, \parencite[7]{HI4}. 
The high-level demands of the customer introduces the supplier to the problem, that the customer wants solved. This allows the supplier to better understand the detailed requirements.

In this case, the customer wanted to eliminate the bookkeeping for all reviews and comments, and assess the students' performance of each handin and review.

Next is the tasks the customer wants performed for the system.
This is normally written in work areas, so clients of the system has tasks and the super users of the system has tasks, \parencite[8-13]{HI4}.
Each task consist of a starting point, ending point and a frequency of appearance, which means someone normally starts the task (also listed) and it will end at some point.
To trace each requirement, a static requirement ID has been inserted.
This way requirements can be moved and changed, but the reference will remain the same.
A task normally requires several small step.
This is listed as subtasks and variants for the task. 
These are normally simple matters and does not need greater attention but they will provide an overview for the customer for how a task will be executed.
This is normally a place errors will be spotted by the customer.
\\
\\
Next is all data types listed, \parencite[14-17]{HI4}.
These provide an overview for the customer and explains what will be saved and transferred in the system and when it is used.
The customer often has a lot experience with the system and will sometimes know what is needed to accomplish a task and from this see, whether more data should be transferred from A to B.
\\
\\
Many times the system can trigger a functionality based on time and dates or a certain input from sensors or devices.
All of these events are listed along with the rules for them to be triggered, \parencite[18]{HI4}.

Unless the requirements are for a brand new idea and no relation to anything, the system normally integrates with other systems \parencite[19-20]{HI4}. 
In this case the STADS system and the LDAP protocol which is used to collect login information and course data for the students. 
Since this is external systems, documentation must be provided and listed how often the data is updated.

The system may also build upon an existing solution which require the customer to deliver information of this, \parencite[21]{HI4}.
However, in this case there is no existing system to handle it.
\\
\\
A big issue for customers is the security of their new system, \parencite[22]{HI4}.
Here the system uses the LDAP protocol to access the student current login and authorize the users to the system.
\\
\\
The design of what the customer sees, the graphical user interface, are sometimes very important, but was not an issue with this system, \parencite[23]{HI4}.
Documentation for how to use the system or install was also a minor issue, \parencite[24]{HI4}, since the user is an IT developer himself.

Response time and maintenance, \parencite[25-27]{HI4} is listed with all acceptable values for the system in and off peak-time, when it is available and how it is maintained.
\\
\\
Lastly, an explanation of how the elicitation process was performed is described.



\end{document}