%---------------------------------------------------------------------
% Preamble for Handin6 in Apa6 style
%---------------------------------------------------------------------

%---------------------------------------------------------------------
% Apa6 style preamble
%---------------------------------------------------------------------

\usepackage[american]{babel}
\usepackage[utf8]{inputenc}

\usepackage{csquotes}
%\usepackage[T1]{fontenc}              



\input{../Modules/SubfilesSetup}


%BibLatex
\usepackage[style=apa, sortcites=true, sorting=nyt, backend=biber]{biblatex} 
\DeclareLanguageMapping{american}{american-apa} %BibLatex mapping
\addbibresource{References.bib}

\title{Final report}
\shorttitle{Handin 6 in TISPIT}

\author{Rasmus Bækgaard and Lasse Brøsted Pedersen}

\affiliation{Aarhus University}

\leftheader{Bækgaard and Pedersen}


\abstract{This paper introduces the reader to assignments done in the course TISPIT made by Group 6.
An SRS the group has written for the course ITDMAT will be discussed, how SRSs can be better and more correct, how SRSs differ written in use case-style and task-description style and what differs from these two. 
A formal specification of has been rewritten using symbolic names and special syntax to modify existing behavior.
Finally the groups SRS are attached with all modifications received after a review. }

\keywords{Handin 6, Task-Description versus use cases, SRS writing, Formal specification adjustment } %shown after abstract