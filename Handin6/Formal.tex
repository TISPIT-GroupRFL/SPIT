\documentclass[Main]{subfiles}

\begin{document}

\section{Formal specification}
%This part shall contain the version of the selected formal VDM-SL specification with the requested extension. Any relevant articles from Handin2 should also be referenced here. 
%New content: Reflection over the value of using formal specification shall be added (around 1/2 page).

See attachment \textit{Handin3.pdf} for "Hand-in 3: Understanding and extending a formal specification" \parencite[]{HI3}.
\\


Employing a formal specification for a system entails creating a mathematical model of a system. Creating a model helps clarify requirements and make any assumptions explicit. This is important since assumptions can cause misunderstandings, that may not be caught before development of the system. In creating a formal model, system defects can be detected, and fixed. A formal model of a system is unambiguous, and can be mechanically checked for inconsistencies. Models can be analysed, and used to prove that a system based on this model will have some desired property. Creating a model will not however, guarantee that the requirement specification is complete.

Formal methods can used to precisely specify the behaviour of a system without designing it, this is also the goal of the technique described in the article \parencite[]{Specifying}. In the article, a technique for specifying system requirements is introduced. The technique includes using symbolic names, and a special syntax for describing various types of system behaviour, and thus closely resembling formal methods. The author of that article lists similar advantages from the described techniques as formal methods. 

Because formal models take time and money to create, and not always provide a benefit, models are mostly used for mission-critical or high security parts of a system. In the case of the system modelled in handin 3\parencite[1-5]{HI3}, the key/lock system for hotel rooms are a high security system, and thus formally modelling such a system can be used to prove to the customer, that the lock system for the rooms is secure, which can be very valuable. The customer may not always be able to understand a formal model, so specifying a system using formal models may not always be feasible.




\end{document}