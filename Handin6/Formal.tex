\documentclass[Main]{subfiles}

\begin{document}

\section{Formal specification}
%This part shall contain the version of the selected formal VDM-SL specification with the requested extension. Any relevant articles from Handin2 should also be referenced here. 
%New content: Reflection over the value of using formal specification shall be added (around 1/2 page).

Employing a formal specification for a system entails creating a mathematical model of a system. Creating a model helps clarify requirements and make any assumptions explicit. A formal model of a system is unambiguous, and can be mechanically checked for inconsistencies. Models can be analysed, and used to prove that a system based on this model will have some desired property. Because formal models take time and money to create, and not always provide a benefit, models are mostly used for mission-critical or high security parts of a system. In the case of the system modelled in handin 3\parencite[1-5]{HI3}, the key/lock system for hotel rooms are a high security system, and thus formally modelling such a system can prove ...


- clarify requirements
- assumptions
- defects
- specify but not design
- prove system properties
- useful for mission critical parts or high security

ref til artikel - artikel beskriver næsten formal specification, så den der skulle kunne drages paralleller..


\end{document}