\documentclass[Main]{subfiles}
\begin{document}


\subsection{Functional requirements}

Functional requirements may be tasks to support, data to store, systems to integrate, etc. 
Here is a fictitious example without relation to the present delivery. 
The functional requirement is that the system must support a number of tasks, including Section \ref{sec:ManageStudents}, and preferably eliminate the current problems.

\subsection{Submit review (Example)}

\fboxsep0pt
\colorbox{light-gray}{\begin{minipage}{\textwidth}
\begin{DataIntro}
\tStart{When the student has finished a review for an assigned handin.}
\tEnd{When the review has been submitted. }
\tFreq{Depends on the number of reviewers for each exercise, and frequency of exercises.}
\tUsers{Students}
\end{DataIntro}
\end{minipage}}


\fboxsep0pt
\colorbox{light-gray}{\begin{minipage}{\textwidth}
\begin{TaskTable}

\Record{Submit Submit review file(s) and record comment, assessment(okay/not okay) and timestamp}{}{}
\RecordAddi
{a}
{Review has already been submitted: update review and associated data.}{}

\Record{Distribute review to reviewees.}{The system could distribute the handin to the reviewee when passing review deadline.}{}

\RecordAddi
{p}
{\textbf{Problem:} Students send reviews as email to reviewees and teachers. This approach is error-prone.}{}{}
\end{TaskTable}
\end{minipage}}

The table lists the subtasks and problems of task C5. 
There is thus a requirement to support each of the table lines to some extent. 
The data about Start, End and Frequency are not in the table, meaning that they are not requirements.

The requirement lines are numbered. 
Variants of a requirement line are marked with letters a, b, etc. 
In the example, the supporter may record the request (subtask 2) or find the request if it has been recorded already (variant 2a). 
Problems relating to a requirement line are marked with the letters p, q, etc. 
A cross reference to a subtask, a variant, or a problem will look like this:
	\textit{See C5-2 or See problem C5-2p}\fxnote{Correct this}.
\\
\\
\textbf{Requirement}. 
Column 1 of a table specifies the customer's demand, e.g. a subtask the system must support, or a problem it should eliminate. 
\\
\\
\textbf{Solution}. 
Column 2 specifies the system's support. 
The column may show the customer's current imagination of a solution -- if he has one. 
This is not a requirement or a wish, but only a possible solution to help the supplier understand the demand. 
In many cases the field will be empty. 
In the reply, the supplier will fill in the solution he proposes .
\\
\\
\textbf{Code}. 
Column 3 may be used in different ways depending on the nature of the project. 
The customer may use it for requirement priorities before sending the request for proposal, or give a score for the supplier's solution when he assesses it. 
Another possibility is that the supplier has to fill in column 3 with a code that specifies the delivery.
\\
\\
\textbf{Text outside tables}
In the example, Start, End, Frequency, and Users are outside the table. 
They are not requirements, but assumptions the supplier can make. 
In general, text outside the tables can serve several purposes:

\begin{enumerate}[A]
\item \textbf{Assumptions} behind the requirements, for instance that the task must be supported for this kind of users, this frequency of use, etc.

\item \textbf{Requirement notes} that elaborate column 1 in the table. 
In principle they should be inside the table, but they don't fit well. 
One example is a list of access rights to the system.

\item \textbf{Solution notes} that elaborate column 2 in the tables. 
They are not requirements but example solutions. 
One example is various ways a user can look up a code in a table.

\item \textbf{Examples} and other information to help the reader understand the requirements.
\end{enumerate}

\end{document}