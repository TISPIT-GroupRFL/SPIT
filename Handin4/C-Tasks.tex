\documentclass[Main]{subfiles}
\begin{document}

\chapter{C-Tasks/Tasks to support}\label{cha:C}
The system must support all user tasks in this chapter, including all subtasks and variants, and mitigate the problems. The subtasks are numbered for reference purposes. They don't have to be carried out in this sequence, and many of them are optional. The user decides what to do and in which sequence. A subtask may also be repeated during the same task.

\section{Work area: Student management}
This work area comprises adding and removing students from the system.

\begin{tabular}{l  p{13cm}}
 \textbf{User profile:} & \textbf{Lecturer.} All lecturers are experienced IT users and have administrative rights to the system. \\
 \textbf{User profile:} & \textbf{Student.} All students are experienced IT users and use the system to form groups.  \\
 \textbf{Environment:} & Office or similar.
\end{tabular}

\subsection{Manage students}\label{sec:ManageStudents}

\begin{DataIntro}
\tStart{On course start, when forming/modifying groups, when students go inactive.}
\tEnd{When the students are correctly recorded, and groups are as desired.}
\tFreq{Only needed on course start, and a few times during the first week, unless many students drop out during the course. About 30-50 students per course.}
\tUsers{Lecturers}
\end{DataIntro}


\begin{TaskTable}
\Record{add students}{import from CN group + manually (stads)}{}

\Record
{remove students}
{•}
{}

\Record
{pull students out of group}
{•}
{}

\Record
{add student to group}
{•}
{}


\RecordAddi
{a}
{Aktion to take}
{Expectation}
{}

\RecordAddi
{b}
{Aktion to take}
{Expectation}
{}

\Record
{create group}
{•}
{}

\Record
{delete group}
{•}
{}

\Record
{exclude student from review list}
{•}
{}

\Record
{freeze groups}
{•}
{}
\end{TaskTable}

\subsection{Form groups}

\begin{longtable}{l p{13cm}}
 \textbf{Start:} & On course start, when not in a group, or when adding group members. \\
 \textbf{End:} & When the the groups are as desired.  \\
 \textbf{Frequency:} & Performed at least once by all students, but seldom more than once. \\
 \textbf{Users:} & Students \\
\end{longtable}

\begin{itemize}
\item Form a group
\item Invite other students to the group.
\item Send out invite notice
\item Join a group
\end{itemize}

\section{Work area: Submission management}
This work area comprises of adding exercises, updating exercises, assessing and possibly overriding submission and review status for groups.

\begin{tabular}{l  p{13cm}}
 \textbf{User profile:} & \textbf{Lecturer.} All lecturers are experienced IT users and have administrative rights to the system. \\
 \textbf{User profile:} & \textbf{Student.} All students are experienced IT users and use the system to submit handins and reviews, and monitor their own status.  \\
 \textbf{Environment:} & Office or similar.
\end{tabular}

\subsection{Manage exercise}

\begin{longtable}{l p{13cm}}
 \textbf{Start:} & When exercises needs to be inspected, added, updated or removed. \\
 \textbf{End:} & When the exercise(s) have been inspected, added, updated or removed.  \\
 \textbf{Frequency:} & About one exercise added weekly. Other variants are rare. \\
 \textbf{Users:} & Lecturers \\
\end{longtable}

\begin{itemize}
\item Inspect exercise
\item Add exercise (variant: update)
\item Record description, deadlines (initial, review, resubmission), number of reviewers, timestamp
\item Remove exercise
\item Notify students
\item freeze reviews / submission
\item Nice to have: versioning
\end{itemize}

\subsection{Manage submission and review status}

\begin{longtable}{l p{13cm}}
 \textbf{Start:} & When status and/or submissions of the groups needs to be assessed or altered. \\
 \textbf{End:} & When the When status and/or submissions have been managed.  \\
 \textbf{Frequency:} & Several times a day, depending on the exercise schedule. \\
 \textbf{Users:} & Lecturers \\
\end{longtable}

\begin{itemize}
\item Assess submission + review status for groups
	\begin{itemize}
	\item \textbf{Current problem:} students send reviews as email to lecturer - time consuming to keep track of.
	\item \textbf{Possible solution:} The system provides an overview, showing all groups, and status (submission ,feedback and review) for each exercise. Filter  
	\end{itemize}
\item Read feedback, reviews and submissions, including older versions
\item Override group status
\end{itemize}

\subsection{Submit handin}

\begin{longtable}{l p{13cm}}
 \textbf{Start:} & When the group has finished a handin for an exercise. \\
 \textbf{End:} & When the handin has been submitted.  \\
 \textbf{Frequency:} & Atleast once per group per exercise. Depends on the frequency of the exercises. \\
 \textbf{Users:} & Students \\
\end{longtable}

\begin{itemize}
\item Submit handin file
\item Record comment and timestamp
\end{itemize}

\subsection{Submit review}

\begin{longtable}{l p{13cm}}
 \textbf{Start:} & When the student has finished a review for an assigned handin. \\
 \textbf{End:} & When the review has been submitted.  \\
 \textbf{Frequency:} & Depends on the number of reviewers for each exercise, and frequency of exercises. \\
 \textbf{Users:} & Students \\
\end{longtable}

\begin{itemize}
\item Submit review file(s)
\item Record comment, assessment(okay/not okay), and timestamp
\end{itemize}


\end{document}