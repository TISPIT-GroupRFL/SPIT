\documentclass[Main]{subfiles}
\begin{document}

\chapter{Tasks to support}\label{cha:C}
The system must support all user tasks in this chapter, including all subtasks and variants, and mitigate the problems. 
The subtasks are numbered for reference purposes. 
They don't have to be carried out in this sequence, and many of them are optional. 
The user decides what to do and in which sequence. 
A subtask may also be repeated during the same task.



\section{Work area: Student management}
This work area comprises adding and removing students from the system.

\begin{tabular}{l  p{13cm}}
 \textbf{User profile:} & \textbf{Lecturer.} All lecturers are experienced IT users and have administrative rights to the system. \\
 \textbf{User profile:} & \textbf{Student.} All students are experienced IT users and use the system to form groups.  \\
 \textbf{Environment:} & Office or similar.
\end{tabular}



\subsection{Manage students}\label{sec:ManageStudents}

\begin{DataIntro}
\tStart{On course start, when forming/modifying groups, when students go inactive.}
\tEnd{When the students are correctly recorded, and groups are as desired.}
\tFreq{Only needed on course start, and a few times during the first week, unless many students drop out during the course. About 30-50 students per course.}
\tUsers{Lecturers}
\tID{001}
\end{DataIntro}

\begin{TaskTable}
\Record{Add students to the list of students in the system.}{List of enrolled students can be imported from STADS.}{}
\RecordAddi
{p}
{\textbf{Problem:} Currently a spreadsheet is made manually.}{}{}

\Record
{Remove students from the system.}{}{}

\Record
{Pull students out of group.}{}{}

\Record
{Add students to group.}{}{}

\Record
{Create student group.}{}{}

\Record
{Delete student group.}{}{}

\Record
{Exclude student from review list.}{}{}
\RecordAddi
{p}
{\textbf{Problem:} Inactive students may be assigned reviews.}{}{}

\Record
{Freeze student groups, preventing students from modifying the groups.}{}{}
\end{TaskTable}





\subsection{Form groups} \label{sec:FormGroups}

\begin{DataIntro}
\tStart{On course start, when not in a group, or when adding group members.}
\tEnd{When the the groups are as desired.}
\tFreq{Performed at least once by all students, but seldom more than once.}
\tUsers{Students}
\tID{002}
\end{DataIntro}

\begin{TaskTable}
\Record{Form a group.}{}{}
\Record{Invite other students to the group.}{}{}
\Record{Send out invite notice.}{}{}
\Record{Join a group.}{}{}
\end{TaskTable}






\newpage

\section{Work area: Submission management}
This work area comprises of adding exercises, updating exercises, assessing and possibly overriding submission and review status for groups.

\begin{tabular}{l  p{13cm}}
 \textbf{User profile:} & \textbf{Lecturer.} All lecturers are experienced IT users and have administrative rights to the system. \\
 \textbf{User profile:} & \textbf{Student.} All students are experienced IT users and use the system to submit handins and reviews, and monitor their own status.  \\
 \textbf{Environment:} & Office or similar.
\end{tabular}




\subsection{Manage exercise} \label{sec:ManageExercise}

\begin{DataIntro}
\tStart{When exercises needs to be inspected, added, updated or removed.}
\tEnd{When the exercise(s) have been inspected, added, updated or removed.}
\tFreq{About one exercise added weekly. Other variants are rare.}
\tUsers{Lecturers}
\tID{003}
 
\end{DataIntro}

\begin{TaskTable}
\Record{Inspect exercise.}{}{}
\Record{Add exercise and record associated data as defined in section \ref{sec:exercise}.}{}{}
\RecordAddi
{a}
{This exercise has already been added: update exercise and record updated associated data as defined in section \ref{sec:exercise}. }{Nice to have: Exercise versioning}{}

\Record{Remove exercise and associated handins and reviews.}{}{}
\Record{Notify students using mail or similar.}{}{}
\Record{Disable submission of handins and reviews.}{}{}
\end{TaskTable}





\subsection{Manage group submission status}\label{sec:ManageGroupSubmissions}

\begin{DataIntro}
\tStart{\code{When} status and/or submissions of the groups needs to be assessed or altered.}
\tEnd{\code{When} status and/or submissions have been managed. }
\tFreq{Several times a day, depending on the exercise schedule}
\tUsers{Lecturers}
\tID{004}
\end{DataIntro}
\begin{TaskTable}
\Record
{Assess handin submission + feedback status for groups.}
{The system provides an overview, showing all groups, and status (handin submissions and feedback) for each exercise. 
Groups with negative feedback could be flagged to enhance overview.}{}

\RecordAddi
{p}
{\textbf{Problem:} Students send reviews as email to lecturer - time consuming to keep track of.}{}

\Record
{Assess handin and review submission status along with feedback for individual students.}
{The system provides an overview, showing all students, and status (handin and review submissions and feedback) for each exercise.}
{}

\Record
{Read feedback, reviews and handin submissions, including older versions.}
{}
{}

\Record
{Override group status, such as marking handins as accepted, while noting the reason.}
{}
{}
\end{TaskTable}





\subsection{Handle end of lecturing}\label{sec:EOLecturing}

\begin{DataIntro}
\tStart{When the deadline for reviews of the last exercise of the semester has been passed.}
\tEnd{When the reports has been extracted. }
\tFreq{Once per quarter.}
\tUsers{Lecturers}
\tID{005}
\end{DataIntro}

\begin{TaskTable}
\Record{Generate a list of students whom have failed to deliver any handins or review.}{}{}

\Record{Generate a list of students whom have delivered all handins and reviews.}{}{}

\Record{Generate a list of students whom have had any status overridden along with associated notes.}{}{}

\end{TaskTable}



\newpage
\subsection{Inspect individual submission status}\label{sec:ManageStudentSubmissions}

\begin{DataIntro}
\tStart{\code{When} status and/or submissions of the student needs to be inpected or altered.}
\tEnd{\code{When} status and/or submissions have been managed. }
\tFreq{Several times a day, depending on the exercise schedule}
\tUsers{Students}
\tID{006}
\end{DataIntro}

\begin{TaskTable}
\Record{Inspect personal handin and review submission status along with feedback.}{The system provides an overview, showing all exercises, and status (handin and review submissions and feedback) for each exercise. The system may also show reviewees for each exercise when the handin deadline has passed.}{}

\Record{Read feedback, reviews and handin submissions.}{}{}

\end{TaskTable}





\subsection{Submit handin}\label{sec:SubmitHandin}

\begin{DataIntro}
\tStart{When the group has finished a handin for an exercise.}
\tEnd{When the handin has been submitted. }
\tFreq{Atleast once per group per exercise. Depends on the frequency of the exercises.}
\tUsers{Students}
\tID{007}
\end{DataIntro}

\begin{TaskTable}

\Record{Submit handin and comments.}{}{}
\RecordAddi
{a}
{Handin has already been submitted: update handin and comment.}{}

\Record{Distribute handin to reviewers.}{The system could assign reviews and distribute the handins to the reviewers when passing the deadline for submissions.}{}

\RecordAddi
{p}
{\textbf{Problem:} Students send handins as email to reviewers, after consulting the table of reviewers. This approach is error-prone - students may misread the review assignments, and send the handin to the wrong students, or even directly to the lecturer.}{}


\end{TaskTable}




\newpage
\subsection{Submit review}\label{sec:SubmitReview}

\begin{DataIntro}
\tStart{When the student has finished a review for an assigned handin.}
\tEnd{When the review has been submitted. }
\tFreq{Depends on the number of reviewers for each exercise, and frequency of exercises.}
\tUsers{Students}
\tID{008}
\end{DataIntro}

\begin{TaskTable}

\Record{Submit review file(s) and record comment, assessment(okay/not okay) and timestamp}{}{}

\RecordAddi
{a}
{Review has already been submitted: update review and associated data.}{}

\Record{Distribute review to reviewees.}{The system could distribute the handin to the reviewee when passing review deadline.}{}

\RecordAddi
{p}
{\textbf{Problem:} Students send reviews as email to reviewees and teachers. This approach is error-prone - students may misread the review assignments, and send the handin to the wrong recipient. Most of all this approach means the lecturer will receive a large number of reviews from students, which leads to a lot of bookkeeping.}{}{}
\end{TaskTable}



\end{document}