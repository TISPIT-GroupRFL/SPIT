\documentclass[Main]{subfiles}
\begin{document}

\section{Availability}

The system is out of operation when it doesn't support some of the users as usual. 
The cause of the breakdown may be: 

\begin{enumerate}

\item The customer's issues, e.g. errors in the customer's equipment.

\item External errors, e.g. power failure.

\item The supplier's issues, e.g. errors in software or configuration.

\item Planned maintenance.

\item Insufficient hardware capacity.

\end{enumerate}
\textbf{Solution note: Measuring availability}\\
A breakdown is counted as at least 20 minutes, even if normal operation is resumed before. 
If the following period of normal operation is less than 60 minutes, it is considered part of the breakdown period.
\\
\\
When the supplier is not responsible for operations, only breakdowns with cause 3 are included in the availability statements. 
When the supplier is responsible for operations too, he is also responsible for causes 2, 4, and 5.
\\
\\
The operational time in a period is calculated as the total length of the period minus the total length of the breakdowns for which the supplier is responsible. 
The availability is calculated as the operational time divided by the total length of the period. When only some of the users experience a breakdown, the availability may be adjusted. 
One way is to calculate the availability for each user and take the average for all users. 

\begin{DynTable}{Availability requirements}

\Doc
{The availability must be calculated periodically. 
The calculation should compensate for the number of users experiencing breakdowns. }
{The availability is stated monthly and calculated as described above.}
{}

\Doc
{In the period from 4:00pm to 12:00pm on Friday, the system must have high availability.}
{In these periods the total availability is at least $\underline{\quad}$\%. \\
(The customer expects 99.5\%)}

\Doc
{In other periods the availability may be lower.}
{In these periods the total availability is at least $\underline{\quad}$\%. \\
(The customer expects 99\%)}
{}


\end{DynTable}

\end{document}