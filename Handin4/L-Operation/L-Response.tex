\documentclass[Main]{subfiles}
\begin{document}

\section{Response times}
It is important that response is so fast that users are not delayed. 
Response time is particularly important during the busiest hours, the peak load periods, which are friday from 4-12 pm.
\\
\\
When the system is operating, it must be able handle the number of transactions specified below with the specified response time. 
The figures are estimated from task frequency (Chapter C), data volumes (Chapter \ref{cha:D}) and statistics from the present operation about peak load periods. 
The figures are the nominal load, i.e. the supplier is not responsible for response time if the actual load exceeds the nominal load.

\textbf{Nominal load}
\begin{enumerate}

\item Simple queries in clinical sessions (C.10): 10 per second on average.

\item Updates in clinical sessions (C.10): 2 per second on average.

\item  Simple queries in patient management (C1 to C4): 3 per second on average.

\item Public web access: 5 page loads per second on average.

\end{enumerate}

Solution note: Measuring response time
The response time is the period from the user sends his command to the result is visible and the user can send a new command. A command means a key press or a mouse click. All measurements are made in peak load periods with the actual number of users, assuming that the actual load is within the nominal load above.

Production work: Measurements are made with a setup according to Chapter G.

The public web part: Measurements are made on a PC connected to the Internet through a 1 MB connection with low traffic on the route to the servers, but with peak load of the servers themselves.


\end{document}