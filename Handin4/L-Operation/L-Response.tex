\documentclass[Main]{subfiles}
\begin{document}

\section{Response times}
It is important that response is so fast that users are not delayed. 
Response time is particularly important during the busiest hours, the peak load periods, which are friday from 4-12 pm.
\\
\\
When the system is operating, it must be able handle the number of transactions specified below with the specified response time. 
The figures are estimated from task frequency (Chapter \ref{cha:C}), data volumes (Chapter \ref{cha:D}) and statistics from the present operation about peak load periods. 
\\
\\
\textbf{Solution note: Measuring response time}\\
The response time is the period from the user sends his command to the result is visible and the user can send a new command. 
A command means a key press or a mouse click. 
All measurements are made in peak load periods with the actual number of users, assuming that the actual load is within the nominal load above.

\begin{DynTable}{Response time requirements}
\Doc
{\textbf{Fractile}. The time specified below must apply in almost all cases.}
{In any one-hour period, $\underline{\quad}$ \% of the response times must be within the limits. (The customer expects 98\%.)}
{}

\Doc
{Response time measurements must be made regularly in the peak load periods.}
{Measurements are made once a week with a stop watch.\\
Or: The system measures all the time.}
{}

\Doc
{When moving from one field to the next, the user's typing speed must not be slowed down.}
{Typing is possible within $\underline{\quad}$ s.\\
(The customer expects 0.2 s.)}
{}

\Doc
{When moving from one screen to the next, data must be visible and typing possible within the mental switching time (around 1.3 s).}
{Data is visible and typing possible within $\underline{\quad}$ s.\\
(The customer expects 1.3 s.)}
{}

\Doc
{Lookup in drop-down lists must allow selection from the list within the mental switching time.}
{Selection is possible within $\underline{\quad}$ s.\\
 (The customer expects 1.3 s.)}
{}

\Doc
{Reports used frequently must be visible within the mental switching time.}
{The report must be visible within $\underline{\quad}$ s. \\
(The customer expects 1.3 s.)}
{}

\Doc
{Reports used occasionally must be visible before the user loses patience.}
{The report must be visible within $\underline{\quad}$ s. \\
(The customer expects 20 s.)}
{}

\Doc
{Login must be completed before the user loses patience.}
{The user can start working within $\underline{\quad}$ s. 
In addition to the time he spends typing name and password. (The customer expects 10 s or better.) }
{}

\Doc
{Repeated login when the user temporarily has left the system must be completed before the user loses patience.}
{The user can start working within $\underline{\quad}$ s. In addition to the time he spends typing his identification. (The customer expects 4 s.)}
{}

\end{DynTable}	


\end{document}