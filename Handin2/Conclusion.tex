\documentclass[Main]{subfiles}

\begin{document}

\section{Conclusion}
%generelt
The three articles provides a good view for how to write certain parts of system requirements, what pitfalls some sections have and how to prevent these from occurring.
Although they have different approaches how this should be solved, they provided a method for the requirements from the customers.

%Specifying
The traditional way is by making a template for all aspects of the requirement to ease the process. 
With good templates all requirements will thoroughly be completed, looking at the input, output, validation and handling of the data.
The stakeholders will have a complete system of which will, if done correctly, make it very easy to implement.
However, should mistakes have been introduced or conflicts overlooked the system can be very hard to change.


%Goal
Writing goals for the system and specify them clearly, will map requirements from a high level to the low, technical level. 
This way the stakeholders can verify all goals will be completed.
It will take less time to write the document than the old way, but takes longer to implement.
It will, however, be easier to modify the code should conflicts or mistakes be discovered later on.

%Task
Taking the traditional approach of functional requirements takes 10 times as much work, when all stakeholders has to review and comment the requirements. 
Instead of specifying how work should be done, it is enough to state the work must be done -- later, in the design process, it can be delegated.
This will  allow a smoother process for all the stakeholders, since data-specific requirements will be written in another document.

Using tasks to make requirements is being favored among developers, since the user's requirements are easier to understand and write to system requirements.
About half of the developer teams uses techniques like these and it may very well be the future way of developing new systems.








\end{document}