\documentclass[Main]{subfiles}

\begin{document}

\section{Comparison}
The three articles tackles the challenges of writing specification requirements in different ways but do also have similarities.
The first article  is written with high level goals for the system at hand, the second is focused on specifying functions on detail and the third article mainly focused on the functional requirement.


The first article on goals insures the right requirements are specified in relation to the stakeholders needs. 
Similar to the third article about task description, there is an overall goal/task to accomplish and smaller goals/tasks required to be fulfilled.
Goals and tasks relates in the structure of way they are satisfied -- A goal/task can be determined as complete in a graph by looking at the underlying goals/tasks.

The author of the first article also states the requirements should be traceable from top-level to the technical-level and written so the requirements can be used in the development\cite[p. 3]{Goal}.
Similar to the second articles author who makes sure all requirement get fulfilled throughout the system with functions, macros and events.
\\
\\
Over time different approach of writing system requirements appears.
The second article from 1980\cite{Specifying} is very implementation oriented since the wording is very programming inspired; it has input and output, uses triggers and macros and states that all inputs are processed by functions.

The third article, on the other hand, states a task must be fulfilled, but does not specify how it should be done.
This can be defined later in the process and will take a short time compared to older methods.
This is mainly due to the way stakeholders must approve the requirements -- instead of specifying Process A must be done by a computer and Process B must be done by a human, the process must be done and who should do it will be specified later on.
Compared to the second article this is many times faster, since stakeholders won't have to approve the data-specific in the requirements.
This is also the case compared to the first article where the requirements goals can be traced all the way to the technical level -- however it not nearly as long as the second articles takes to write all in-, outputs and functions.




%foreksel
priority of a goal/task is 



%\begin{itemize}[G]
%\item Goal specificerer meget præcist
%\item Goal graph og subtasks minder om hinanden
%\item Minder lidt om Diskret matematik
%\item Grafer med mål
%\item Mål har prioritet
%\item Goal attributes -- en kvalitativ værdi, der kan løse konflikter
%\end{itemize}
%
%
%\begin{itemize}[S]
%\item Speci består af templates man bruger til alting
%\item Templates angiver ikke UC'er -- angiver input og output.
%	\begin{itemize}
%	\item Hvordan påvirker dette systemet -- eg. en viser
%	\item Angives med funktioner
%	\item Macro minder om en if-else
%	\item Macro kan trigger et event
%	\item Angives i SR
%	\end{itemize}
%\item Meget specifikt hvad der skal ske med input og output (formelt)
%\end{itemize}
%
%
%\begin{itemize}[T]
%\item Task ny metode at udvikle krav -- ingen UC'er men tasks (meget lidt specifike)
%\item Task kører i bestemte mønstre
%\item Task virker kun på funktionelle krav -- derfor ingen datakrav
%\item SuperTask og subtasks. AND/OR complete
%\item Subtask skal udføres, men hvordan er underordnet 
%\item Hurtigt at lave og nemmer at forstå for kunden
%\item Kun afprøvet én gang (2003) -- og frem til 2011\dots
%\item Ingen overskrifter på "standard måden" (minus abstract)
%\end{itemize}
%
%
\end{document}
