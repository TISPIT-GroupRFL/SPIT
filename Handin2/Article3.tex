\documentclass[Main]{subfiles}

\begin{document}

\section{Task Description As Functional Requirements}
%\textbf{Introduction}\\
The author's purpose for writing this article is to describe how a new method the author (Søren Lauesen) and Marianne Mathiassen has developed for specifying functional requirements, produces higher quality requirements which are faster to specify and easy to verify and validate.
The reason for developing a new method for specifying functional requirements, is that traditional functional requirements has some handicaps, which the article addresses. 

The main problem about traditional functional requirements is, according to the article, that these specify the system's role but ignore the system's context. 
Such traditional requirement leads to prematurely division of work between the computer and the user, which can cause serious problems, and that is the issue the author is addressing with his new method\cite{Task} (Task-Based Descriptions).
\\
\\
The author uses facts about use cases to show how Task-Based Descriptions differ from traditional functional requirements.  
In addition to that, he uses experience and knowledge to support his method and the advantages it gives. 
He also use the data and feedback he gained from real cases where he applied his solution, to support his conclusions.

The author states that traditional functional requirements separate the work between the computer and the user, and that it is a problem his method can handle better. 
He give examples of why traditional functional requirements have problems and how his method can do it better, but he does not really say if that is always the case. 
It sounds like he assumes traditional requirements always will have problems their method can handle better, without stating why that always would be the case.
\\
\\
%\textbf{Main text}\\
The key concept in this article is to understand how the author's new method (task based functional requirements) is structured and how it works. 
When you understand that, you have the information you need to understand the author's reasoning.

The task based functional requirements first states the work area of the requirement, which gives background information and states the overall purpose. 
After that individual tasks are described containing purpose, trigger/preconditions, frequency, critical situations, subtask and variants. 
The concept is, that the task specifies what the user and computer must do together. That is done via imperative language (for example "Find room"), which hide whether a human or a computer carries out the subtask.  
\\
\\
As mentioned this new method avoids splitting the labor between the computer and the user, which the author sees as a big advantage. 
Splitting the labor can cause big problems according to the author, because a lot of spe\-cification has to be changed, if it is discovered late in a project that a requirement cannot be met by the specified actor.
Another important point about this method, is that it can save developers and customers a lot of time, because requirements in task-form is easier to understand and write.

A real case he studied is presented in the article, where a hospital is going through a tender process. 
Afterwards he used his new method to handle the hospitals requirements. 
He found that his method reveals critical requirements that traditional methods easily overlook, but that separate data descriptions are usually necessary.
\\
\\
If developers and customers adopts the author's new method, it will result in functional requirements structured in a whole new way, which developers has to be trained to do.
According to the author it will help both developers and customers speeding up the specification phase a lot. 
Moreover it helps reveal critical requirements and make functional requirements of high quality, which saves both customers and suppliers money.

If the new method gets widely accepted, but some people refuse to adopt the method, they would of cause get a hard time in tender processes, and eventually have to integrate this new method.
The author did a quick survey of how many teams uses similar techniques in 2012\cite{Task2}, and concluded about 45\% of the teams used it.
\\
\\
%\textbf{Conclusion}\\
The author's conclusion to his method is that it brings many advantages that contribute to making higher quality functional requirement faster. 
The method makes the requirements easy to understand for the customer and the requirements are specified and reviewed much faster than with the traditional method.
It does require some expert training, before time will be saved with this method though. 
Moreover it is made clear that this method does not specify data and it does not cover quality requirements, and have to be specified separate.

\end{document}