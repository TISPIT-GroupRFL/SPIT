\documentclass[Main]{subfiles}

\begin{document}
\section{Goal-Oriented Requirements Engineering: A Guided Tour}

%\textbf{Introduction}\\
%Purpose
The author wishes to explore the technique of goal-oriented requirements specification.
In order to convey the author's points there are used examples and a case study.
Additionally there are reviews of research going into other aspects of goal-oriented requirements engineering.

%Question
The main reason for writing this article is to compare previously used requirements specification techniques.
Additionally the author wishes to make a link between requirements and goals to technical details -- this is done to increase traceability.

%Information
Another positive effect is increased readability because the Goals describe what the stakeholders need.

%Assumption
Goals are essential when creating correct requirements in line with project stakeholder's wishes.
The reason for defining goals in the early phase of a project is to remove focus from existing solutions and instead look at the heart of what the stakeholders need.
\\
\\
%\textbf{Main text}\\
%Concepts
By making a goal graph\cite[section 5.1]{Goal} of goals where they are linked together, a parent goal can have multiple subgoals.
If all subgoals are satisfied and the goal graph is said to be an AND-graph then the parent goal is said to be satisfied.

Additionally there can be OR-graphs of goals that work in a similar manner, but only one of the subgoals has to be satisfied in order for the parent goal to be satisfied.

To aid the creation of goals it can be helpful to define some different goal types.
The article focuses on functional, non-functional and informational to name a few.
Functional goals are the actual services a system is expected to deliver.
Non-functional goals are related to security, safety and performance.

%Point of view
In the authors opinion it is important to specify goals clearly because they should map directly to requirements\cite[p. 3]{Goal}.

%Implications
If the goal-oriented approach of defining requirements get widely used, then the requirements created will be more precise and closer to what the stakeholders really need.
\\
\\
%\textbf{Conclusion}\\
%Conclusion
Using Goals to derive requirements helps to provide a rational for the requirement.
If a goal graph is developed, it is possible to trace goals from high-level requirements to low-level technical details\cite[p. 11]{Goal}.

\end{document}
