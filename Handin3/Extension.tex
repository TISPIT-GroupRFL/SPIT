\documentclass[Main]{subfiles}

\begin{document}

\section{Extension for request}

The VDM-SL model shall be updated such that locks to a room can be opened with more than one card.
The solution must obey the following rules:
\begin{itemize}
\item All cards must still be unique.
\item The guests must be able to obtain additional cards.
\item Guests may only obtain extra cards to rooms for which they already have at least one card.
\end{itemize}
This problem can be solved in many ways -- in the following a solution is explained.
To allow more than one card to open a room, it must be possible to create at least one additional card with the same keys (first and second) as the original card corresponding to that room.

\subsection{Maintaining \code{Card} uniqueness}

The first step in the solution involves making sure that all cards are still unique even when carrying the same keys. This is important, because it enables the \code{Card}s to be handled using sets, additionally it allows the \code{Card}s to be tracked, should that functionality be desired. 
This is accomplished by adding a card ID to the \code{Card} record as stated in \codeTitle \ref{lst:cardID}.

\begin{vdmsl}[label=lst:cardID,caption= \code{Card} with ID ]
Card ::	cardId 	 : CardId
				fst 	   : Key  
				snd 	   : Key; 
\end{vdmsl}
The uniqueness of all \code{Card} records is enforced by the adding the following to the invariant of the \code{Hotel} state:

\begin{vdmsl}[label=lst:hotelInv,caption= Invariant added to \code{Hotel} ]
forall c1, c2 in set {c | c in set dunion rng h.guests} & c1.cardId <> c2.cardId
\end{vdmsl}
To help create new, unique \code{CardId}s the following implicit operation is added:

\begin{vdmsl}[label=lst:cardIDop,caption= Helper operation for unique \code{CardId}s ]
GetNewCardId() cId:CardId

ext rd guests : map Guest to set of Card

post cId not in set {c.cardId | c in set dunion rng guests};
\end{vdmsl}
The operation listed in \codeTitle \ref{lst:cardIDop} has been added to all instances of the constructor for \code{Card} as follows:
\\ \code{mk_Card(GetNewCardId(),-,-)}.

\subsection{Adding \code{Card}s}

The important part of the solution is enabling the guests to obtain a copy of a \code{Card} already possessed by the \code{Guest}. 
The obvious solution is to copy a \code{Card}, and add it the guest's set of \code{Card}s. This can be accomplished by the operation listed in \codeTitle \ref{lst:copyOp}.

\begin{vdmsl}[label=lst:copyOp,caption= Operation \code{AddCopyOfCard()} ]
AddCopyOfCard(g:Guest,old_c:Card)

ext wr guests : map Guest to set of Card 
    
pre g in set dom guests and 
		old_c in set guests(g)	

post let new_c = mk_Card(GetNewCardId(), old_c.fst, old_c.snd) in 
		 new_c in set guests(g) and 
		 guests(g) = guests~(g) union {new_c};
\end{vdmsl}
The precondition of \code{AddCopyOfCard()} states that the \code{Guest} \code{g} must be checked in to the \code{Hotel}, and the \code{Guest} \code{g}, must currently own the card to be copied (\code{old_c}).
The postcondition states that a new \code{Card}, \code{new_c} must be added to the \code{guest}'s set of \code{card}s, and \code{new_c} must have a unique \code{CardId} and the same keys as \code{old_c}.

The operation \code{AddCopyOfCard()} requires the {guest} to bring the {card} to be copied, and allows a {guest} to copy a {card} even if the corresponding \code{Room} has been checked into by another {guest}, thus rendering that {card} useless. 
To remedy these shortcomings, the following operation has been added:

\begin{vdmsl}[label=lst:extraOp,caption= Operation \code{AddExtraCardForRoom()} ]
AddExtraCardForRoom(g:Guest,r:Room)

ext rd desk   : Desk
    wr guests : map Guest to set of Card
    
pre g in set dom guests and  
		exists ca in set guests(g) & ca.snd = desk.prev(r)

post let old_c:Card be st old_c.snd = desk.prev(r) and old_c in set guests(g) in 
		let new_c = mk_Card(GetNewCardId(), old_c.fst, old_c.snd) in 
		new_c in set guests(g) and 
		guests(g) = guests~(g) union {new_c}; 
\end{vdmsl}
The operation listed in \codeTitle \ref{lst:extraOp} takes a \code{Guest}, \code{g}, and the \code{Room}, \code{r}, for which the guest desires an additional \code{Card}. \\ 
The precondition states that the \code{Guest} must be checked in to the \code{Hotel}, and the must posses a \code{Card} with the key most recently checked into \code{Room r}. This means that only the \code{Room}'s current \code{Guest} can obtain an additional \code{Card} for that \code{Room}.
\\
The postcondition states that a new \code{Card} (\code{new_c}) must be created with a unique \code{CardId} and the same keys as the \code{Card} corresponding to \code{Room r}. This \code{Card}, \code{new_c} must be added to the \code{Guest}'s set of \code{Card}s. 

\end{document}
